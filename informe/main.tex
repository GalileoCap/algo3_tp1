%&pdflatex
\documentclass[12pt]{article}
\usepackage[margin=1.0in]{geometry}

\usepackage{g-util, g-uba, g-algo}
\usepackage{subfiles}

\usepackage{tikz}

\renewcommand{\figurename}{Fig.}

\begin{document}

\titulo{TP1: Técnicas Algorítmicas}
%\subtitulo{Técnicas Algorítmicas}
\fecha{21 de Septiembre de 2022}
\materia{Algoritmos y Estructuras de Datos III}
\integrante{Cappella Lewi, Federico Galileo}{653/20}{glewi@dc.uba.ar}
\integrante{Mallol, Martín Federico Alejandro}{208/20}{martinmallolcc@gmail.com}
\integrante{Teplizky, Gonzalo Hernán}{201/20}{gonza.tepl@gmail.com}
\integrante{Stemberg, Uriel Nicolás}{213/20}{uri.stemberg@gmail.com}
\maketitle

\newpage
\begin{abstract}
  En este trabajo se resuelven tres ejercicios de la cátedra usando distintas técnicas algorítmicas para explorar espacios de soluciones, buscando encontrar un método que sea eficiente. \\
  \textbf{Palabras clave}: \textit{Fuerza Bruta}, \textit{Backtracking}, \textit{Greedy}, \textit{Programación Dinámica}, \textit{Complejidad Temporal}, \textit{Complejidad Espacial}.
\end{abstract}

\tableofcontents

\newpage
\subfile{ejercicio1.tex}

\newpage
\subfile{ejercicio2.tex}

\newpage
\subfile{ejercicio3.tex}

\end{document}
