%&pdflatex
\documentclass[12pt]{article}
\usepackage[margin=1.0in]{geometry}

\usepackage{g-util, g-uba, g-algo}
\usepackage{subfiles}

\usepackage{algorithm, algorithmic}

\begin{document}

\titulo{TP1: Técnicas Algorítmicas}
%\subtitulo{Técnicas Algorítmicas}
\fecha{21 de Septiembre de 2022}
\materia{Algoritmos y Estructuras de Datos III}
\integrante{Cappella Lewi, Federico Galileo}{653/20}{glewi@dc.uba.ar}
\integrante{Mallol, Martín Federico Alejandro}{208/20}{martinmallolcc@gmail.com}
\integrante{Teplizky, Gonzalo Hernán}{201/20}{gonza.tepl@gmail.com}
\integrante{Stemberg, Uriel Nicolás}{213/20}{uri.stemberg@gmail.com}
\maketitle

\newpage
\begin{abstract}
  Para este trabajo el enfoque será puesto en experimentar con distintas técnicas algoritmicas recursivas, con el fin de encontrar algoritmos más rápidos y eficientes que aquellos que emplean fuerza bruta. Mas allá de que en la mayoría de los casos sea imposible evitar una complejidad exponencial, se intentará acotar lo más posible el tiempo de cómputo de cada programa implementado en este trabajo. Se deben resolver tres ejercicios donde, cada uno de ellos se tratará sobre el comportamiento de una de estas tres técnicas en específico: backtracking, algoritmo goloso/greedy/miope (cualquiera de estas tres definiciones es válida), y por último, programación dinámica. \\
  \textbf{Palabras clave}: \textit{Fuerza Bruta}, \textit{Backtracking}, \textit{Greedy}, \textit{Programación Dinámica}, \textit{Complejidad Temporal}, \textit{Complejidad Espacial}.
\end{abstract}

\tableofcontents

\newpage
\subfile{ejercicio1.tex}

\newpage
\subfile{ejercicio2.tex}

\newpage
\subfile{ejercicio3.tex}

%\newpage
%\begin{thebibliography}{1}
%
%\bibitem{blabla} https://www.blablabla.com/blabla/
%\bibitem{blabla} https://www.blablabla.com/blabla/
%\bibitem{blabla} https://www.blablabla.com/blabla/
%
%\end{thebibliography}

\end{document}
