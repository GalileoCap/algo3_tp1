%&pdflatex
\documentclass[./main.tex]{subfiles}

\begin{document}

\section{Ejercicio 1: Robots On Ice}
\label{sec:ej1}

\subsection{Presentación}
\label{sec:ej1-intro}

\paragraph{} La primer consigna plantea resolver el problema \textit{UVa 1098, Robots On Ice}\footnote{\url{https://onlinejudge.org/index.php?option=onlinejudge&Itemid=8&page=show_problem&problem=3539}}. En el que se busca contar todos los caminos posibles dentro de un mapa que cumplan ciertas condiciones.

\paragraph{} El mapa es una grilla rectangular de tamaño \(n \times m\), con \(2 \leq n, m \leq 8 \in \mathbb{N}\). Y los caminos empiezan en la posición \((0, 0)\) y terminan en la posición \((0, 1)\), teniendo que pasar en orden por tres posiciones, o check-ins, en momentos equidistantes de el camino. \\
Tanto \(n, m\), y los check-ins, son parámetros de entrada que el programa lee del standard input. Y resultado se devuelve por el standard output. \\
\indent Dentro del mapa sólo es posible moverse en las cuatro direcciones cardinales, sin poder repetir posiciones previamente pisadas.

\paragraph{} Para resolver este problema utilizamos \textbf{backtracking}, armando a fuerza bruta todos los caminos posibles, descartando los que no cumplan con las condiciones pedidas, y contando los que sí. \\
\indent El algoritmo empieza en la posición inicial dada, y prueba moverse en cada dirección. Si el movimiento fue legal, cumple con las restricciones, y no rompe las restricciones a futuro (ver \textbf{\ref{sec:ej1-podas} Podas}), entonces continúa recursivamente, ahora probando moverse desde esta nueva posición. \\
\indent Como para cada posición se prueban cuatro movimientos, la complejidad temporal es \(\bigO{4^{n*m}}\).

\subsection{Podas}
\label{sec:ej1-podas}

\paragraph{} Como parte del algoritmo, implementamos tres tipos de podas para descartar caminos:
\subparagraph{Legales}
\begin{itemize}
  \item El movimiento no se sale del mapa. Osea, se mueve a una posición \((i, j)\) con alguna de las dos coordenadas fuera de rango).
  \item No se pasó ya por el casillero de destino. Para esto llevamos registro de los casilleros por los que se pasó en una matriz de \(n \times m\) booleanos.
\end{itemize}

\subparagraph{Restrictivas}
\begin{itemize}
  \item El casillero de destino. Como los check-ins tienen que ser visitados en tiempos equidistantes necesitamos que en el paso \(\dfrac{n*m*i}{4}\) estemos en el \(i\)-ésimo check-in (para \(0\leq i \leq 4 \in \mathbb{N}\), contando a las posiciones \((0, 0)\), y \((0, 1)\) como check-in 0 y 4 respectivamente).
\end{itemize}

\subparagraph{Preventivas}
\begin{itemize}
  \item Como sólo nos podemos mover de a un casillero por paso y sólo en las 4 direcciones, nos queda que entre las posiciones \((i, j)\) y \((l, k)\) se tienen mínimo \(|l-i| + |k-j|\) pasos. Por lo que nos fijamos que estemos a menos de \(|l-i| + |k-j|\) pasos del paso necesario para el check-in \((k, j)\) (con \((i, j)\) la posición actual). %TODO: Mencionar distancia Manhattan
  \item TODO: Cortar a la mitad
\end{itemize}

%TODO: Comparar tiempos sin diferentes podas, y mencionar tienpo del juez

\end{document}
